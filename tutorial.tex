\documentclass{article}

\usepackage{csquotes}
\usepackage[
    examples=expex,
    font=cochineal,
    trees=forest,
    biblatex=false,
    tipa
]{lingtex}
\usepackage{kantlipsum}

% \RequirePackage{forest}
% \useforestlibrary{linguistics}
% \forestapplylibrarydefaults{linguistics}
% \forestset{
% 	nice trees/.style={
% 		for tree={
% 			parent anchor=south,
% 			child anchor=north,
% 			align=center,
% 			base=top,
% 			inner sep=1pt,
% 			l-=4ex,
% 			before typesetting nodes={% based on nice empty nodes - page 52 of the manual, used in Jason Zentz's answer: http://tex.stackexchange.com/a/216103/
% 				if content={}{
% 					for parent={
% 						for children={anchor=north},
% 						calign=fixed edge angles,
% 						calign angle=60,
% 					},
% 					shape=coordinate,
% 					calign=fixed edge angles,
% 					calign angle=60,
% 				}{},
% 			},
% 		},
% 	},
% 	somewhat nice trees/.style={% needed to avoide divide by 0 errors: http://tex.stackexchange.com/q/204094/42880
% 		for tree={
% 			parent anchor=south,
% 			child anchor=north,
% 			align=center,
% 			base=top,
% 			inner sep=1pt,
% 			l-=4ex,
% 			before typesetting nodes={% based on nice empty nodes - page 52 of the manual, used in Jason Zentz's answer: http://tex.stackexchange.com/a/216103/
% 				if content={}{
% 					for parent={
% 						for children={anchor=north},
% 						calign=fixed edge angles,
% 						calign angle=60,
% 					},
% 					inner sep=0pt,
% 					edge path={\noexpand\path [\forestoption{edge}] (!u.parent anchor) -- (.south)\forestoption{edge label};}, %from http://tex.stackexchange.com/a/281546
% 					calign=fixed edge angles,
% 					calign angle=60,
% 				}{},
% 			},
% 		},
% 	}
% }

% \def\type#1{\textsf{#1}}
% \def\e{\type{e}}
% \def\t{\type{t}}
% \def\ann{\textbf{ann}}
% \def\bill{\textbf{bill}}
% \def\bea{\textbf{bea}}
% \def\cam{\textbf{cam}}
% \def\di{\textbf{di}}
% \def\jan{\textbf{jan}}
% \DeclareMathOperator{\seated}{\textbf{seated}}
% \DeclareMathOperator{\near}{\textbf{near}}
% \DeclareMathOperator{\sing}{\textbf{sing}}
% \DeclareMathOperator{\like}{\textbf{like}}

\def\movesrc{\ensuremath{\circ}}

\title{Ling\TeX}
\author{Jackson Petty}
\date{\today}

\def\ipademo{f@"nEtIks}

\begin{document}
\maketitle

\kant[1]

What is the meaning of \den{this}? Let's look at some IPA examples: \ipa{\ipademo} and \allo{\ipademo} and \phon{\ipademo}. Lexical items like \lex{this} are very \lex[\textsc{ing}]{tse-} useful.

\ex
This is a test
\xe

\ex
\begin{forest} nice trees,
	[CP []
	[\&P
	[TP
	[DP\\Mary, name=specDP1]
	[
	[T]
	[\emph{v}P
	[\movesrc,name=specTP1]
	[
	[\emph{v}]
	[VP, name=VP2
	[]
	[
	[V\\likes]
		[$\star$, name=m2]
	]
	]
	]
	]
	]
	]
	[
	[\&\\but]
	[TP
	[DP\\I, name=specDP2]
	[
	[T]
	[\emph{v}P
	[\movesrc,name=specTP2]
	[
	[\emph{v}]
	[VP, name=VP1
	[]
	[
	[V\\hate, name=lowV]
		[$\star$, name=m1]
	]
	]
	]
	]
	]
	]
	]
	]
	]
	\draw[->] (specTP2) to[out=south west,in=south] (specDP2);
	\draw[->] (specTP1) to[out=south west,in=south] (specDP1);
	\node[draw, below left=15pt and -15pt of lowV, align=center] (dp) {DP\\ our new professor};
	\draw[->, dashed] (dp) to[out=north, in=south] (m1);
	\draw[->, dashed] (dp) to[out=north, in=south] (m2);
	\draw[dotted, thick] ([xshift=-30pt]VP1) arc[start angle=130,end angle=90,radius=3cm];
	\draw[dotted, thick] ([xshift=-30pt]VP2) arc[start angle=130,end angle=90,radius=3cm];
\end{forest}
\xe

\end{document}